\section{Robotic Modeling}
\subsection{Approach}

The general equation of motion for a robotic manipulator can be expressed as:

\begin{equation}
  \label{eom-manipulator}
  M(q)\ddot{q} + C(q,\dot{q})\dot{q} + G(g) = \tau
\end{equation}

\noindent where $M(q)\in\mathbb{R}^{n\times n}$ is the mass matrix of the joints,
$C(q,\dot{q})\in\mathbb{R}^{n\times n}$ is the Coriolis matrix,
$G(q)\in\mathbb{R}^{n}$ is the conservative force vector acted on the arm by
gravity, and $\tau\in\mathbb{R}^{n}$ are the torques commanded to each joint.
Each of the terms on the left hand side of the equation can be directly modeled
when the manipulator joint information is known.
However, when this information is not available, a neural network can be used to
estimate the value of these matrices.

To model the conservative force vector, a robotic manipulator can be controlled
with a simple PD controller and brought to a stop at a given configuration value
$q$.
When the manipulator is at rest, Eq. (\ref{eom-manipulator}) simplifies to:

\begin{align}
  \label{eom-manipulator-stopped}
  M(q)*0 + C(q,0)*0 + G(q) &= \tau\\
  G(q) &= \tau
\end{align}

\noindent which allows the loss function for the function approximator
$\hat{G}(q)$ to be expressed as:

\begin{equation}
  \label{loss-function-gq}
  \hat{G}(q) - \tau = \delta_{Loss}
\end{equation}

To model the mass matrix with a function approximator $\hat{M}(q)$, we must
again model a loss function to train the network.
This proves more difficult, as the mass matrix must be isolated on the left hand
side of the equation.
However, if the number of acceleration samples taken at a given configuration
$q$ satisfy $rk(M(q)) = n_{samples}$, we could then construct an acceleration
matrix $\ddot{Q} = [\ddot{q_{1}}, \ddot{q_{2}}\dots\ddot{q_{n}}]$ which
satisfies $rk(\ddot{Q}) = rk(M(q)) = n$.
Properties of linear algebra now guarantee that a unique inverse of this matrix
must exist, and the equation can then be expressed as:

\begin{align}
  \label{eom-manipulator-accel}
  M(q)\ddot{Q} + C(q,\dot{Q})\dot{Q} + G(Q) &= \mathrm{T}\\
  \label{eom-manipulator-accel-m}
  M(q) &= \ddot{Q}^{-1}(\mathrm{T} - G(Q) - C(q,\dot{Q})\dot{Q})
\end{align}

\noindent where $G(Q)\in\mathbb{R}^{n\times n}$ is a one dimensional matrix
where each column satisfies $G_{i}(Q) = G(q)$.

The problem with this method is that the Coriolis term is still in place, and we
have no estimate for this value.
However, if we take our $\ddot{q}$ samples at the initial point when we first
begin accelerating, we should see that $\dot{q} \approx 0$, and thus Eq.
(\ref{eom-manipulator-accel-m}) can be expressed as:

\begin{align}
  M(q) = \ddot{Q}^{-1}(\mathrm{T} - G(Q))
\end{align}

\noindent which now allows for a loss function to be formulated.

For the Coriolis matrix, it directly depends on the mass matrix $M(q)$, and thus
once an estimate for $M(q)$ is formulated, the Coriolis matrix can be derived
from the given values of the $\hat{M}(q)$ matrix.

\subsection{Experiments}
The experiments for this section will include learning both $\hat{G}(q)$ and
$\hat{M}(q)$ using the proposed method above in two settings: One in a friction
free environment, and another when static friction is present.
The friction free case will be used primarily as a base case, and the accuracy
of both the $\hat{G}(q)$ and $\hat{M}(q)$ will be compared to their true values.
This will serve to see how effective this method is in general, as
friction begins complicating the process.

The frictional case will be further broken down into two sections.
The first will consist of learning the two maps without static friction
compensation, and the second will include static friction compensation.
Static friction will be modeled as a constant $\mu_{s}$, so this can easily be
added into the above formulations (primarily for the mass matrix case).
