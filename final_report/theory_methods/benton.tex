\subsection*{Mass Matrix Modeling}

The general equation of motion for a robotic manipulator can be expressed as:

\begin{equation}
  \label{eom-manipulator}
  M(q)\ddot{q} + C(q,\dot{q})\dot{q} + G(g) = \tau
\end{equation}

\noindent where $M(q)\in\mathbb{R}^{n\times n}$ is the mass matrix of the joints,
$C(q,\dot{q})\in\mathbb{R}^{n\times n}$ is the Coriolis matrix,
$G(q)\in\mathbb{R}^{n}$ is the conservative force vector acted on the arm by
gravity, and $\tau\in\mathbb{R}^{n}$ are the torques commanded to each joint \cite{spong}.
Each of the terms on the left hand side of the equation can be directly modeled
when the manipulator joint information is known.
However, when this information is not available, a neural network can be used to
estimate the value of these matrices.

To model the mass matrix with a function approximator $\hat{M}(q)$, we must
model a loss function to train the network.
This proves more difficult, as the mass matrix must be isolated on the left hand
side of the equation.
However, if the number of acceleration samples taken at a given configuration
$q$ satisfy $rk(M(q)) = n$, we could then construct an acceleration
matrix $\ddot{Q} = [\ddot{q_{1}}, \ddot{q_{2}}\dots\ddot{q_{n}}]$ which
satisfies $rk(\ddot{Q}) = rk(M(q)) = n$.
Properties of linear algebra now guarantee that a unique inverse of this matrix
must exist, and the equation can then be expressed as:

\begin{align}
  \label{eom-manipulator-accel}
  M(q)\ddot{Q} + C(q,\dot{Q})\dot{Q} + G(Q) &= \mathrm{T}\\
  \label{eom-manipulator-accel-m}
  M(q) &= \ddot{Q}^{-1}(\mathrm{T} - G(Q) - C(q,\dot{Q})\dot{Q})
\end{align}

\noindent where $G(Q)\in\mathbb{R}^{n\times n}$ is a one dimensional matrix
where each column satisfies $G_{i}(Q) = G(q)$.

The problem with this method is that the Coriolis term is still in place, and we
have no estimate for this value.
However, if we take our $\ddot{q}$ samples at the initial point when we first
begin accelerating, we should see that $\dot{q} \approx 0$, and thus Eq.
(\ref{eom-manipulator-accel-m}) can be expressed as:

\begin{align}
  M(q) = \ddot{Q}^{-1}(\mathrm{T} - G(Q))
\end{align}

\noindent which now allows for a loss function to be formulated.
This method assumes that the gravity vector is known, or that a close approximate may be made to cancel this term on the right hand side of the equation.
