\section{Perception and Path Trajectory}
\subsection{Uncertainty in Configurations}
After developing the model of mass approximation, the entirety of the robot model URDF will be created. The robot configuration $q_1$, $q_2$, ..., $q_n$ determines how the robot will interact with the task space with the end-effector.  The setup and interaction of the robot is usually modelled using pure values of the configuration.  In real life, many variables can cause uncertainty in the configuration state, like improper installation, introducing friction, noise, etc.  This uncertainty can produce a butterfly effect if the robot has many degrees of freedom, causing the end-effector's placement and orientation to be out of spec.  A model of uncertainty in this robot will be introduced using a noise variable with a Gaussian distribution.  A Kalman filter will be applied to each joint to help mitigate the noise present, which gives the model more accuracy.  As a result, the culmination of each joint filter means that controlling the robot to accomplish a task with the end-effector can be done more confidently.  The output of the Kalman filter will be fed into the controller design that governs the motion necessary to complete tasks.

\subsection{Perception of the End-Effector}
Assuming that the end-effector has a camera and can understand its surroundings, the perception through said camera will be used to achieve motion and interaction with an object in the task space.  A combination of RRT path planning algorithm, Kalman filtering, and Forward/Inverse kinematics will lead to the configuration that the robot is in to have the end-effector reach the object.  As the end-effector moves through space, it perceives its surroundings with some uncertainty about where it is, which the Kalman filter will mitigate.  The robot will need to generate a path to the object, and the forward/inverse kinematics will dictate how the robot achieves that path.

\subsection{Approach}
To begin, some feasibility coding will be used on some basic, low DOF robots to prove that joint filtering is possible.  Two joint types will be analyzed: revolute and prismatic.  Then, implementation of the mass estimation will be combined with the joint filtering.  Also, outputs of the Kalman filter will have to be tested with the controller to test desired outcomes.

Next, the end-effector perception problem requires many disciplines of robotics.  First, a landmark based Kalman filter will be applied to the robot.  This will test our ability to move to the object.  Then, applying kinematic theory to RRT path planning will be used to develop a trajectory of the robot to achieve the end-effector location and orientation based on the Kalman filter readings.